% !TEX TS-program = xelatex
% !TEX encoding = UTF-8 Unicode
% !Mode:: "TeX:UTF-8"

\documentclass{resume}
\usepackage{zh_CN-Adobefonts_external} % Simplified Chinese Support using external fonts (./fonts/zh_CN-Adobe/)
% \usepackage{NotoSansSC_external}
% \usepackage{NotoSerifCJKsc_external}
% \usepackage{zh_CN-Adobefonts_internal} % Simplified Chinese Support using system fonts
\usepackage{linespacing_fix} % disable extra space before next section
\usepackage{cite}

\begin{document}
\pagenumbering{gobble} % suppress displaying page number

\name{郑湛东}

\basicInfo{
  \email{zzdsakura@gmail.com} \textperiodcentered\ 
  \phone{(+86) 158-100-58975} 
  }
 
\section{\faGraduationCap\  教育背景}
\datedsubsection{\textbf{北京林业大学}, 北京}{2016 -- 至今}
\textit{在读本科生}\ 网络工程专业, 预计 2020 年 7 月毕业

\section{\faGithubAlt\ 工具项目}
\datedsubsection{\textbf{zorm}}{\url{https://github.com/Victordong/zorm}}
\role{Golang, Linux}{个人项目}
\begin{onehalfspacing}
学习于gorm的一个golang orm
\begin{itemize}
  \item curd实现
  \item 事务功能实现
  \item 根据业务需要内置假删除
\end{itemize}
\end{onehalfspacing}

\datedsubsection{\textbf{easy api}}{\url{https://github.com/Victordong/async_easyapi}}
\role{Python, Linux}{团队项目}
\begin{onehalfspacing}
  基于salalchemy flask asyncio 提供了基于类,快速curd后端api构建的功能
\begin{itemize}
  \item 简化curd逻辑
  \item 封装事务
  \item 封装自己的错误处理
\end{itemize}
\end{onehalfspacing}

\section{\faUsers\ 商业项目}

\datedsubsection{\textbf{北京林业大学督导系统后端}}{\url{https://github.com/Victordong/bjfu_supervision_back}}
\role{Python, Linux}{团队项目}
\begin{onehalfspacing}
校内督导给教师课程进行评价的系统

工作
\begin{itemize}
  \item 完成大部分后端编写任务, 参与数据库表的设计
  \item 基于mongodb实现便于更新的问卷表结构,易于存储表的填写结果
  \item 基于kafka降低后端模块间的耦合
  \item 基于redis存储计算数据降低系统负载
\end{itemize}
关键词

flask, mysql, mongodb, redis, kafka
\end{onehalfspacing}

\datedsubsection{\textbf{自动配肥机后端}}{\url{}}
\role{Golang, Linux}{团队项目}
\begin{onehalfspacing}
配肥机的服务端, 用于管理肥料数据, 计算配肥价格及元素含量

工作
\begin{itemize}
  \item 完成部分后端编写任务, 参与数据库表的设计
  \item 和前端协商实现自己的url参数处理,简化代码量
  \item 使用阿里接口, 实现短信业务
\end{itemize}
关键词

gin, mysql, 阿里相关服务
\end{onehalfspacing}

\datedsubsection{\textbf{智能终端跟踪系统}}{\url{}}
\role{Python, Linux}{团队项目}
\begin{onehalfspacing}
用于跟踪洒水车和农业喷药无人机的数据管理和监控

工作
\begin{itemize}
  \item 使用自己团队的easyapi 编写完成部分后端任务, 参与数据库表的设计
  \item 使用jinja渲染模版,和前端一起实现了一个较为好用的运行报告渲染系统
  \item 使用redis和captcha实现了一个api可用的验证码生成校验
\end{itemize}
关键词

easyapi, mysql, redis
\end{onehalfspacing}


\section{\faCogs\ IT 技能}
% increase linespacing [parsep=0.5ex]
\begin{itemize}[parsep=0.5ex]
  \item 编程语言: Golang Python C
  \item 平台: Linux
  \item Web后端知识: 熟练使用flask gin 进行后端开发, Mysql相关知识, Redis相关知识, 了解并简单使用过kafka nginx 
  \item 计算机网络: 熟练掌握HTTP协议, 了解TCP/IP协议, Socket编程 和 常见的并发模型
  \item 计算机科学常识: 掌握常见的算法以及数据结构, 了解unix操作系统及其环境编程
\end{itemize}



\section{\faInfo\ 其他}
% increase linespacing [parsep=0.5ex]
\begin{itemize}[parsep=0.5ex]
  \item GitHub: https://github.com/Victordong
  \item 语言: 英语 - CET-4
\end{itemize}

%% Reference
%\newpage
%\bibliographystyle{IEEETran}
%\bibliography{mycite}
\end{document}
